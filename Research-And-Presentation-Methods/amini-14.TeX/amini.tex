\documentclass{book}
\begin{document}
\begin{flushright}
\texttt{CHAPTER ONE}
\hspace*{0.5cm}
\textbf{14}
\end{flushright}
\vspace*{0.7cm}
presence" (Anderson, Rourke, Garrison, & Archer, 2001) that we wanted to apply in a
real-life context. We used the Net to scour the ERIC database and Google (a search
engine) to search for related terms like \emph{peer moderators} and \emph{peer teaching}, and we ordered
texts not available in our university library using online interlibrary loan request forms.
We created a research plan and shared it with a colleague for critical review. We then
downloaded and completed the research ethics forms from our faculty Web page and, of
course, submitted them electronically. Upon approval of the project, we drafted a letter of
introduction to students, in which we informed them of the intent of the research and the
proposed activities. We emailed this letter and opened a forum on a conferencing system
for discussion of the research process. In some cases, a follow-up email was required, but
eventually all eighteen students gave their consent to participate. We then developed a
short Net-based survey on the elements of teaching presence. These results were triangulated
with information from a transcript analysis. During the six weeks of the experiment,
we emailed each of the students reminding them of the day they were to complete
the weekly online questionnaire. After the course completed, we conducted semi-struc-tured
telephone interviews with a sample of the students, applied our transcript analysis
instrument with two independent coders, and reflected on our own experiences of the
course. From these data sources, we drafted and revised a paper and emailed it to the
students for comments(as a member check). after a final revision, we submitted the paper to
the \emph{Journal of Interactive Media in Education}(http://www.jime.ac.uk)-a non-blind, peer-
reviewed, online journal. The article was reviewed by three reviewers and after some
minor edits and improvements, it was accepted for publication by editor. In addition,
we posted the paper along with additional output from our research group on our own
Web dissemination site at http://www.atl.ualberta.ca.
\\\hspace*{0.5cm}Was this e-research example? Certainly the context and the site of investigation
were based on the Net. We used the Net extensively to support data collection and
administration of the project. For example, we conducted our literature review almost
exclusively on the Net, used a conference to archive ongoing discussion with students,
used email to obtain informed consent and to communicate, developed and administered
a Web-based survey, and used the Web in a number of ways for dissemination of results.
However, we weren't dogmatically committed the Net. we used the telephone for
interviews, as not all students had IP(Internet Protocol)-based telephony.
\\\hspace*{0.5cm} In this example,the Web was used in two common applications of e-research.
First,it enabled and made more efficient the process of research practice as a means to research
and disseminate the results. Second, the Net allowed us to investigate an educational
activity taking place on the Net. Rapid communications with subjects throughout the
course of the research as well as investigation of interaction through transcript analysis
shaped the kind and nature of research process.
\begin{flushleft}
\texttt{SUMMARY}
\hspace*{0.5cm}
\end{flushleft}
\vspace*{0.3cm}
The primary goal o this book is to help the reader understand, appreciate, and
control the underlying economics, operating techniques, and ethical consideration of
e-research. Research has many characteristics and qualities is quality itself. Quality
research addresses important problems and is honed to find solutions to those problems.
It is systematic, transparent, and available to the public. The Net provides us with
\end{document} 